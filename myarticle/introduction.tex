\chapter{はじめに}

音楽は世界中で楽しまれ、その作成方法は技術発展により多様化している。近年注目されている音楽の作成方法としてリミックスと呼ばれるものがある。リミックスは既存の曲に音響操作を加えて新しい曲を作成する方法のことであり、1970年代のディスコの発達とともに世界的に普及した。当時はディスコでのDJによる即興のパフォーマンスにより行われていたが、近年のパソコンなどのデジタル機器の発達によって音楽の作成方法として一般的なものとなった。

しかし、録音や音の編集を行うソフトウェアアプリケーション~(DAW)~の操作がリミックスには必要であるため、音楽作成を円滑に行うためには一定の経験が必要となる。したがって、コンピュータプログラムによるリミックスの補助が音楽作成に役立つと考えられる。また、本研究では、リミックスの方法の一つである音色の変換に注目し、ニューラルネットワークによる変換手法を提案する。

さらに、音色の変換を行うためには、ある楽器の音を異なる楽器の音へ変換する技術が必要である。本研究では、画像のスタイル変換を行うPix2pix~\cite{pix2pix}を音色の変換に応用した。Pix2pixは、ニューラルネットワークにより自然な画像を生成する手法であるGAN~\cite{GAN}を応用した手法である。

そして、本研究では、ギターの単音をハープの単音へと提案モデルを用いて変換する実験を行った。その結果、ほとんどの音で音程を維持したまま音色の変換を行うことに成功した。また、データセットの一部の音のみで学習を行った場合でも、ほとんどの音で音程を維持したまま変換を行うことができた。

なお、本論文では、楽譜作成ソフトのMuseScore\footnote{\url{https://musescore.org/}}を用いて音のデータセットを作成し、音声編集ソフトのAudacity\footnote{\url{https://www.audacityteam.org/}}を用いて音波の波形の画像を取得した。