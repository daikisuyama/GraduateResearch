%必要事項
%1ページにまとめる
%追加すべき事項
% - 結果の概要のまとめ(トピックセンテンス)
% - 追加実験の必要性
% - 和音の変換?
% - 課題のまとめ(改善策)
% - データ拡張:位相をずらす、バッチサイズの調整?
\chapter{まとめ}

本研究での実験の結果、提案手法における生成モデルはギターからハープへと音色の変換を行うには十分な表現力を持つことを確認することができた。ただし、ニューラルネットワークのモデルによる細かい音波の表現、学習時のデータの振幅の乱雑さの加え方、安定したデータセットの作成、の三つの点においては課題が残った。そのため、4分割交差検証を行った際にはその影響が増幅し、音程はほとんどの変換で維持できているものの、ハープの音へと変換することができたのは一部のみであった。

また、本研究では波形の観察による考察を行ったが、より定量的な判定を行う必要がある。具体的には、音程が維持されているかと音色が正しく変換されているかの判定を考慮できると良い。前者についてはフーリエ変換を用いて実装することができるが、後者については考察の余地がある。

これらの原因は、振幅をランダムにすることにより学習が安定しないことと微細な振動を表現可能なニューラルネットワークを構築できていないためと考えられる。後者の場合は画像で用いられるGANでの画像の高解像度化の工夫が適用できると期待される。