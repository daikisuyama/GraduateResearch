%必要事項(全体)
%内容の改善
%図の改善
%現在形と過去形のチェック
%表現の統一
%改ページのチェック
%載せた論文に目を通す
%参考文献の調整

%文字の大きさ、紙の大きさ、ドライバ、エンジン、文書クラス
\documentclass[10pt,a4paper,dvipdfmx,uplatex]{jsreport}

%ページ番号の出力
\pagestyle{plain}

%画像
%draftオプションで枠のみに
\usepackage{graphicx}
%位置指定
\usepackage{float}
%ハイパーリンク
\usepackage[colorlinks=false]{hyperref}
\usepackage{pxjahyper}
%数式
\usepackage{amsmath,amssymb}
%アルゴリズム
\usepackage[boxed,figure]{algorithm2e}
%文字色
\usepackage{color}
%コメントアウト
\usepackage{comment}
%かっこいい表
\usepackage{booktabs}
%タイトルの調整
\usepackage{titlesec}
%括弧の設定
\usepackage{interval}
%IfAppendix
\usepackage{apptools}
%参照の変更
\usepackage{prettyref}
%目次の設定
\usepackage{tocloft}
\usepackage{etoolbox}
%図の設定
\usepackage{caption}
\usepackage[subrefformat=parens]{subcaption}
%余白の設定
\usepackage[top=25mm,bottom=25mm,left=30mm,right=30mm]{geometry}

%タイトルの番号付けの設定
\setcounter{secnumdepth}{3}

%各種タイトルの設定
\titleformat{\chapter}[hang]{\Huge\bfseries}{\IfAppendix{\appendixname\Alph{chapter}}{第\thechapter~章}}{1zw}{}{}
\titlespacing*{\chapter}{0pt}{0pt}{50pt}
\titleformat*{\section}{\LARGE\bfseries}
\titlespacing{\section}{0pt}{15pt}{13pt}
\titleformat*{\subsection}{\Large\bfseries}
\titlespacing{\subsection}{0pt}{12pt}{10pt}
\titleformat*{\subsubsection}{\large\bfseries}
\titlespacing{\subsubsection}{0pt}{9pt}{7pt}

%目次の設定
\renewcommand{\cfttoctitlefont}{\Huge\bfseries}
\renewcommand{\cftchapfont}{\normalsize\bfseries}
\setlength{\cftchapnumwidth}{4em}
\renewcommand{\cftchapleader}{\cftdotfill{\cftchapdotsep}}
\renewcommand{\cftchapdotsep}{\cftdotsep}

%図目次の設定
\renewcommand{\cftloftitlefont}{\Huge\bfseries}
\renewcommand{\cftfigpresnum}{図~}
\setlength{\cftbeforefigskip}{0pt}
\cftsetindents{figure}{1.5em}{5em}
%図目次にchapterを加える(付録は載せない)
\preto\figure{
  \ifnum\value{figure}=0
  \phantomsection
  \IfAppendix{}{
    \addtocontents{lof}{
      \begin{flushleft}
        \normalsize\bfseries{第\thechapter~章}
      \end{flushleft}
      \par
      \vspace{-7.5pt}
    }
  }
  \fi
}

%開き括弧の設定
\intervalconfig{soft open fences}

%参照の変更(prettyref)
\newrefformat{sec}{\ref{#1}節}
\newrefformat{fig}{図\ref{#1}}
\newrefformat{tab}{表\ref{#1}}
\newrefformat{eq}{式\ref{#1}}
\newrefformat{cha}{第\ref{#1}~章}
\newrefformat{app}{\appendixname\ref{#1}}

%argmaxとargminのコマンド
\newcommand{\argmax}{\mathop{\mathrm{arg~max~}}\limits}
\newcommand{\argmin}{\mathop{\mathrm{arg~min~}}\limits}

%付録への図の記載の簡略化コマンド
% #1:パス名
% #2,#3:(図のファイル名、図のcaption)
% #4,#5:(図のファイル名、図のcaption)
% #6,#7:(図のファイル名、図のcaption)
% #8,#9:(図のファイル名、図のcaption)
\newcommand{\appfig}[9]{
  \begin{figure}[htb]
  \centering
  \begin{minipage}{0.24\columnwidth}
  \centering
  \includegraphics[width=0.9\columnwidth]{#1#2}
  \caption[]{#3}
  \end{minipage}
  \begin{minipage}{0.24\columnwidth}
  \centering
  \includegraphics[width=0.9\columnwidth]{#1#4}
  \caption[]{#5}
  \end{minipage}
  \begin{minipage}{0.24\columnwidth}
  \centering
  \includegraphics[width=0.9\columnwidth]{#1#6}
  \caption[]{#7}
  \end{minipage}
  \begin{minipage}{0.24\columnwidth}
  \centering
  \includegraphics[width=0.9\columnwidth]{#1#8}
  \caption[]{#9}
  \end{minipage}
  \end{figure}
}

\begin{document}

%タイトル
\begin{titlepage}
\begin{center}
\vspace*{160truept}
{\huge ニューラルネットワークによる}\\
\vspace{10truept}
{\huge 音色の変換}\\
\vspace{200truept}
{\Large 教養学部後期課程学際科学科総合情報学コース}\\
\vspace{10truept}
{\Large 陶山大輝}\\
\vspace{10truept}
{\Large 学籍番号:08-192021}\\
\vspace{10truept}
{\Large 指導教官:金子知適准教授}\\  
\end{center}
\end{titlepage}

%目次
\setcounter{tocdepth}{2}
\tableofcontents
\phantomsection
\addtocontents{toc}{~\hfill\textbf{Page}\par}
\clearpage

%図目次
\listoffigures
\phantomsection
\addtocontents{lof}{~\hfill\textbf{Page}\par}
\clearpage

%はじめに
\chapter{はじめに}

音楽は世界中で楽しまれ、その作成方法は技術発展により多様化している。近年注目されている音楽の作成方法としてリミックスと呼ばれるものがある。リミックスは既存の曲に音響操作を加えて新しい曲を作成する方法であり、1970年代のディスコの発達とともに世界的に普及した。当時はディスコでのDJによる即興のパフォーマンスとして行われていたが、近年のパソコンなどのデジタル機器の発達により音楽の作成方法として一般的なものとなった。

しかし、音の編集などを行うソフトウェアアプリケーション~(DAW)~の操作がリミックスには必要であり、音楽作成を円滑に行うためには一定の経験が必要となる。したがって、コンピュータプログラムによるリミックスの補助が音楽作成に役立つと考えられる。また、本研究では、リミックスの方法の一つである音色の変換に注目し、ニューラルネットワークによる音色の変換手法を提案する。

そして、本研究では、画像のスタイル変換を行うPix2pix~\cite{pix2pix}を元に作成した提案モデルを用いてギターの単音からハープの単音へ音色を変換する実験を行った。その結果、音の高さを維持したまま変換を行うことができたが、ハープの音を表現できたのは一部の音のみであり、音色の変換におけるいくつかの課題が浮かび上がった。

なお、本論文では、音のデータセットを作成する際に楽譜作成ソフトのMuseScore\footnote{\url{https://musescore.org/}}を利用し、音波の波形画像を作成する際に音声編集ソフトのAudacity\footnote{\url{https://www.audacityteam.org/}}を利用した。

%背景:音楽
%追加すべき事項
% - "音楽の表現"における音声処理の補強
% - "音楽の表現"の式(理解が必要)
% - "音楽の表現"の具体例(論文,既存研究)
% - "音楽の表現"の図
% - "音楽の表現"の単語の説明
\chapter{背景:音楽}

本章では、音の定義を行い、音の表現方法を紹介する。

\section{音の定義}

音とは、弾性体中を伝播する波により起こされる音波が聴覚により感じられるもののことである。

\subsection{音響信号}

音は時間方向の変化量であるため、音響信号と呼ばれる。音響信号は連続的な信号であるが、コンピュータで扱うために離散的な信号へと変換する必要がある。また、変換の際には標本化~(サンプリング)~と量子化が必要である。まず、サンプリングは一定の時間を空けて離散的に測定を行うことであり、1秒あたりのサンプリング回数をサンプリング周波数と呼ぶ。そして、量子化は信号の大きさを離散的に表現することであり、信号の大きさを表現するビット数を量子化ビット数と呼ぶ。

\subsection{楽音}

楽音は周期性のある音波を持つ音のことである。本論文では楽音としての音を生成することを目標とする。また、楽音は長さ、大きさ、高さ、音色の四要素を持つ。

\subsubsection{音の長さと大きさ}

音の長さは音波の時間長により決まり、音の大きさは音波の振幅により決まる~(\prettyref{fig:gakuon1})~。音波の時間長が長いほど音の長さは長くなり、音波の振幅が大きいほど音の大きさは大きい。

\subsubsection{音の高さと音色}

音の高さと音色は直感的にはそれぞれ音波の周期構造の長さと形により決まる~(\prettyref{fig:gakuon2})~。また、音の高さは音波の周波数により決まり、周波数の高い音ほど音の高さは高くなる。そして、音の長さと大きさと高さが同じ時の音の違いを音色と呼び、それぞれの楽器は異なる音色を持つ。

\begin{figure}[b]
\centering
\begin{minipage}{0.48\columnwidth}
\centering
\includegraphics[width=\columnwidth]{figure/gakuon1.png}
\caption{音波}
\label{fig:gakuon1}
\end{minipage}
\begin{minipage}{0.48\columnwidth}
\centering
\includegraphics[width=\columnwidth]{figure/gakuon2.png}
\caption{音波の拡大図}
\label{fig:gakuon2}
\end{minipage}
\end{figure}

%ここで改ページ
\clearpage

\section{音の表現}

音をニューラルネットワークで扱うためには、楽音としての特徴を学習するための適切な表現を得る必要がある。また、本節は~\cite{musictutorial}及び~\cite{DL_ASP}を参考に作成し、本節の図は~\cite{musictutorial}のFigure~4と~\cite{timbretron}のFigure~2を利用している。

\subsection{二次元での表現}

素朴で連続的な音の表現である音響信号は離散的な一次元データへと変換される~(\prettyref{fig:audio_signal})~。音響信号は〇〇などの研究で△△という長所を利用されるが、ニューラルネットワークが楽音としての特徴を学習するには多くのデータが必要になると考えられる。

したがって、次節以降で紹介する二次元データへと変換して扱うことが一般的である。音響信号を周波数成分ごとに分解する操作により、次元としては時間と周波数を選ぶことが多い~(時間-周波数表現)~。また、この変換により音響信号の楽音としての特徴が明確化されるため、より効果的に特徴を学習できると考えられる。

ここで、画像のニューラルネットワークを楽音の二次元データの表現に適用する場合、主に二つの相違点に注意が必要である。一つ目は、局所的な相互関係についてである。画像では近くのピクセル同士の色合いや強度が似ているが、時間-周波数表現では周波数方向での局所的な相互関係が弱い。二つ目は、スケール不変性についてである。画像ではスケールを変化させても特徴は不変であることが多いが、時間-周波数表現ではスケールを変化させることで特徴も変化する。

\begin{figure}[b]
\centering
\includegraphics[width=0.8\columnwidth]{figure/audio_signal.png}
\caption{音響信号}
\label{fig:audio_signal}
\end{figure}

%ここで改ページ
\clearpage

\subsection{STFT}

STFT~(Short~Time~Fourier~Transform)~は時間-周波数表現を生成する基本的な手法であり、中間周波数を利用した線形間隔のバンドパスフィルターを用いて周波数成分を分解する。また、横軸を時間で縦軸を周波数とした画像をSTFTにより生成することができ、これをスペクトログラムと呼ぶ。ここで、STFTの具体的な計算方法については〇〇に詳しい。%勉強する

しかし、人間の聴覚系の周波数分解能は線形ではないかつ楽音の解析のために作成されたわけではないため、STFTのバンドパスは楽音の解析に適しているとは言えない。したがって、STFTはニューラルネットワークで扱う表現の作成において人気のある手法ではない。ただし、STFTは音源分離などに用いられることがある。%研究例を挙げる

\subsection{メルスペクトログラム}

メルスペクトログラムはメル尺度~\cite{melscale}を用いてスペクトログラムにおいて周波数方向の圧縮を行ったものである。また、メル尺度は人間の感じる音の高さの差と差が等しくなるように調整した尺度であり、~\cite{mel}では周波数を$f$として\prettyref{eq:mel}のように定式化される。

\begin{align}
    \label{eq:mel}
    mel(f)=1127.01048\log{(\frac{f}{700}+1)}
\end{align}

メル尺度は人間の聴覚系に合わせた尺度であるため、スペクトログラムよりも楽音の解析に適している。また、経験的にもメルスペクトログラムは楽音のタスクに適していることが言える。%研究例

\begin{figure}[b]
\centering
\begin{minipage}{0.48\columnwidth}
\centering
\includegraphics[width=\columnwidth]{figure/stft.png}
\caption{STFT}
\label{fig:STFT}
\end{minipage}
\begin{minipage}{0.48\columnwidth}
\centering
\includegraphics[width=\columnwidth]{figure/mel.png}
\caption{メルスペクトログラム}
\label{fig:mel}
\end{minipage}
\end{figure}

%ここで改ページ
\clearpage

\subsection{CQT}

CQT~(Constant~Q~Transform)~は対数振幅の中心周波数を使用した二次元の表現である。CQTは対数振幅を用いており、音の高さの分布に近い。したがって、基音を正確に識別する際に用いられ、和音の識別や書き換えに使うことができる。

また、計算量としてはSTFTやメルスペクトログラムより重い。そして、単純な対数振幅のスペクトログラムが有効な例もある。

Rainbowgram

与えられた音の高さの集合におけるエネルギー分布のこと。多くの場合は西洋音楽の12音階をその集合とする。つまり、クロマグラムは周波数方向に畳み込んだCQTと見なすことができる。また、クロマグラムは他の表現方法よりも処理が進んだものなので、それ自身を特徴量として用いることができる。

\begin{figure}[b]
\centering
\begin{minipage}{0.48\columnwidth}
\centering
\includegraphics[width=\columnwidth]{figure/cqt.png}
\caption{CQT}
\label{fig:cqt}
\end{minipage}
\begin{minipage}{0.48\columnwidth}
\centering
\includegraphics[width=\columnwidth]{figure/choroma.png}
\caption{クロマグラム}
\label{fig:chroma}
\end{minipage}
\end{figure}
    
%ここで改行
\clearpage


%ここを提案モデルにもいれよう

\section{音楽の特徴}

一般には音楽で特定の音色への変換を行うことは難しいが、本研究では三つの要素に分解することで単音での音色の変換を音楽へと適用することが可能であると考えた。

\subsection{楽器の重ね合わせ}
    
音楽はそれぞれの楽器から出力される音の重ね合わせになっている。楽器ごとに音色は異なるため、音色変換を行うには楽器ごとの音波に分解すること~(音源分離)~が必要であると考えられる。なお、楽曲の作成時に楽器ごとに分離したデータ~(パラデータ)~として保存することが一般的であるため、パラデータの公開が一般的になれば音源分離の必要はなくなる。

\subsection{音の重ね合わせ}

ある音が重音である場合はそれぞれの単音について音色変換を行う必要がある。本研究では、データセットとして単音のみを使用するが、重音もデータセットに加えることで音色変換が可能であると考えられる。また、この際にデータセットが膨大な量になる可能性があり、追加するデータセットの工夫が必要である。

\subsection{音の繋ぎ方}

楽器ごとの音波に分解可能で重音も表現可能である時、時間方向の音の繋ぎ方を工夫する必要がある。また、単音の変換を一定の単位時間で行うことで、音楽もその単位時間で分割して変換することで可能であると考えられる。ただし、単位時間の定め方により性能が変わると考えられるので、単位時間は慎重に定める必要がある。

%ここで改ページ



%背景:ニューラルネットワーク
\section{ニューラルネットワークとは}
ニューラルネットワークとはニューロンとニューロン間のシナプスによる結合で形成される脳のネットワークを模した数理モデルのことである。入力層と出力層を持ち、シナプスの結合強度を変化させることで問題に最適なネットワークを構成することを目標とする。\par
また、入力層と出力層の間に隠れ層を加えて多層にし層間に活性化関数を用いて非線形分離を行うことで、複雑なネットワークを表現をすることが可能になる。そして、任意の活性化関数が微分可能であれば、誤差逆伝播法により損失関数を高速に求めることができる。\par


\section{ディープラーニングとは}

%ディープラーニングとは
%GANとは
%pix2pixとは
%CycleGAN(検討)
%Wavenet(検討)
%スペクトログラム(検討)

%提案手法
%ここからは提案手法

\section{データ形式}
音のファイル形式としてはWAVを用いる。一般的にはMP3やMP4などの非可逆圧縮形式が広く用いられるが、WAVは波形データを直接保持しており本論文ではWAVを扱いやすいと考えたためである。

%WAVの持つ情報:A,B,CがあるがAは割愛する、みたいな説明

%離散化の際のデジタル信号の単位時間あたりの標本化の回数をサンプリング周波数と呼び、量子化の際のデジタル信号の大きさを表現するビット数を量子化ビットと呼ぶ。

%また、音は時間を定義域としたアナログ信号なので、離散化及び量子化を行ってデジタル信号へと変換することでデジタル機械の扱えるデータ形式となる。

% subsectionでメタ情報の説明
% itemizeでも

音のデータそのものだけでなく量子化ビット,サンプリング周波数,チャンネル数,サンプリング数のメタ情報を得ることができる。

また、サンプリング周波数を44100Hz,量子化ビットを16bit,チャンネル数を1,サンプリング数を44100に固定して今回の実験を行った。

\section{データセット}

楽譜作成ソフトの\href{https://musescore.org/ja}{MuseScore}を利用して国際の階名表記でA0からC8に含まれる半音をwav形式で88音生成した。また、この88音は一般的な88鍵のピアノで出すことのできる全音階の音であり、人間が音程として聞き分けることのできる限界の音域としてこの音域を選んだ。


%他の楽器との間との比較も?
%全く異なる楽器や近い楽器
%どこにかくか、表現
また、楽器としてはエレキギターとハープを選んだ。


\section{判定器}

(1)音程が維持されているか

(2)変換されて音色が変換されているか

波形とスペクトログラム?


%---------

%単音の変換手法


%それができたらどうなるかをここで書く


\section{音色の変換}

%なぜ音色の変換がしたいのか↓
%音色の返還の際には次のような難しさがある
%三つの要素に分解することで音楽の音色の変換を単音の音色の変換に帰着することができる

音色の変換を音楽で行う際に(音の特徴を学習する)が、以下の三つの点を解決するのが難しいと考えられる。また、以下の三つを解決することで、単音における音色の変換を音楽に適用することができる。

\subsection{楽器の重ね合わせ}

楽器ごとに音色が異なるので、楽器ごとの音波に分解して音色変換を行うことが良いと考えられる。なお、楽曲の作成時に楽器ごとに分離したデータ~(パラデータ)~で保存しておけば、直接楽器ごとの音波を利用できる。


\subsection{時間方向の音の繋ぎ方}

%ここは後で
\textcolor{red}{時間方向での音の繋ぎ方は都合の良いように分割していくことでなんとかなるのではないか…?、分割する(1つの音の判定を行う)のは難しい…?、自己回帰?}

\subsection{音の重ね合わせ}

単位時間の楽器の音に注目した時、楽器ごとの音波に分離したとしても和音のようにその単位時間で複数の種類の音が鳴っている場合も難しいと考えられる。

\textcolor{red}{とりあえず試しでも良いので実験を行いたい}

%まとめ
%やったこと、やってないことを書く
\chapter{まとめ}


\section{future work}

%書くならfuture workみたいなところに


%それができたらどうなるか
%future research?
\subsection{音楽の変換}

%なぜ音色の変換がしたいのか↓
%音色の返還の際には次のような難しさがある
%三つの要素に分解することで音楽の音色の変換を単音の音色の変換に帰着することができる

音色の変換を音楽で行う際に(音の特徴を学習する)が、以下の三つの点を解決するのが難しいと考えられる。また、以下の三つを解決することで、単音における音色の変換を音楽に適用することができる。

\subsubsection{楽器の重ね合わせ}

楽器ごとに音色が異なるので、楽器ごとの音波に分解して音色変換を行うことが良いと考えられる。なお、楽曲の作成時に楽器ごとに分離したデータ~(パラデータ)~で保存しておけば、直接楽器ごとの音波を利用できる。


\subsubsection{時間方向の音の繋ぎ方}

%ここは後で
\textcolor{red}{時間方向での音の繋ぎ方は都合の良いように分割していくことでなんとかなるのではないか…?、分割する(1つの音の判定を行う)のは難しい…?、自己回帰?}

\subsubsection{section}{音の重ね合わせ}

単位時間の楽器の音に注目した時、楽器ごとの音波に分離したとしても和音のようにその単位時間で複数の種類の音が鳴っている場合も難しいと考えられる。

\textcolor{red}{とりあえず試しでも良いので実験を行いたい}



\subsection{判定器}

(1)音程が維持されているか

(2)変換されて音色が変換されているか

波形とスペクトログラム?


%自己回帰モデル,スペクトログラム

\subsection{和音の手法}

二音のみの組み合わせでできるのか
ネットワークを変えるべきか
ネットワークをアップデートする必要があるのか
前処理を工夫するか
フーリエ変換を用いてsin波に分解するか

%謝辞
\chapter{謝辞}

本研究を進める際にご指導をして頂いた指導教官の金子知適准教授にまずは厚く感謝を申し上げます。また、研究に関して助言を頂いた金子研の構成員の方々や試問教員の方々にも感謝の意を表します。

%参考文献
\bibliographystyle{junsrt}
\bibliography{reference}

%付録
\appendix
\chapter{}
\label{sec:appendix}

\section{実験時のパラメータ}
\label{sec:appendix_params}

実験時のパラメータを表\ref{tab:params}に示す。

\begin{table}[h]
\label{tab:params}
\caption{}
\begin{center}
    \begin{tabular}{|ll|}\hline
        パラメータ & 値 \\ \hline \hline
        インプットのバッチサイズ & 1 \\ 
        学習でのエポック数 & 1000 \\ 
        学習での学習率 & 0.0002 \\ \hline
    \end{tabular}
\end{center}
\end{table}

\section{データセットの分割方法}
\label{sec:appendix_split}

生成モデルの汎化能力を調べた際のデータセットの分割の仕方を表\ref{tab:split}に示す。

\begin{table}[h]
\label{tab:split}
\caption{}
\begin{center}
    \scalebox{0.8}{
    \begin{tabular}{|lllllllllllllllllllllll|}\hline
        番号  \\ \hline \hline
        0 & a5s & g2 & e6 & f6 & g3 & c1 & a2s & f7 & f5 & c2s & b5 & e4 & b6 & d4s & a4 & b7 & e1 & g4s & f4 & a7s & g1 & f2s\\ 
        1 & a2 & d1 & g1s & g4 & d5s & d6 & a3s & a1 & a5 & c6s & f1s & c1s & a1s & c5 & d2s & g7 & c2 & e7 & b2 & g7s & f5s & d1s \\ 
        2 & e3 & g5 & c7s & d4 & g5s & d6s & g6 & c8 & c4 & c4s & a0s & a3 & d3 & d5 & b0 & a6 & a6s & f7s & c5s & d3s & f3 & d7s \\ 
        3 & b4 & f3s & f6s & e2 & b1 & e5 & f2 & c3 & a4s & f1 & c6 & a7 & d2 & g6s & g2s & g3s & c3s & f4s & a0 & d7 & b3 & c7 \\ \hline
    \end{tabular}
    }
\end{center}
\end{table}

\end{document}