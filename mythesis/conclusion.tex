%追加すべき事項
% - DDSPの簡単な説明
\chapter{まとめ}

本章では、実験結果の考察と展望についてまとめる。

\section{実験結果のまとめ}

提案手法の目標は音の高さと大きさを維持したままでギターの単音からハープの単音への音色の変換を行うことである。実験の結果、上記の目標に足る表現力を持つ提案モデルを作成できたが、汎化能力が十分ではないことも明らかになった。

また、汎化能力の低い理由としては主に二点あると考えられる。一点目は、Pix2pix~\cite{pix2pix}で用いられるDropout層を用いていない点である。二点目は、データ拡張の工夫が振幅の無作為化のみである点であり、時間方向にずらすなどの工夫もあると考えられる。

さらに、本研究では波形の観察を中心とした考察を行ったが、本研究の考察を明確化するためにはより定量的な判定を行う必要があると考えられる。具体的には、音の高さの維持、音の大きさの維持、音色の変換、という三点での定量的な判定を行う必要がある。

\section{展望}

本研究の実験が成功した場合、以下の三点を解決することで音楽の変換に適用できると考えられる。

\subsection{楽器の重ね合わせ}

音楽はそれぞれの楽器から出力される音の重ね合わせになっている。楽器ごとに音色は異なるため、音色変換を行うには楽器ごとの音波に分解すること~(音源分離)~が必要である。また、楽曲の作成時に楽器ごとに分離したデータ~(パラデータ)~の公開が一般的になれば音源分離の必要はなくなる。

\subsection{音の重ね合わせ}

ある音が単音の重ね合わせである場合は単音ごとに分離をして音色変換を行うことが必要である。また、本研究ではデータセットとして単音のみを使用するが、重音もデータセットに加えることで音色変換が可能であると考えられる。しかし、この際にデータセットが膨大な量になる可能性があり、追加するデータセットの工夫が必要である。

\subsection{音の繋ぎ方}

上記の二つが可能である時、時間方向の音の繋ぎ方を工夫する必要がある。研究前は単位時間ごとに区切るのみで可能であると考えたが、この場合はリズムの変更などで単位時間の基準が変わった際に対応することができない。しかし、この課題は二次元の表現を用いることで解決することができると考えられる。

%また、GoogleのMagentaプロジェクトによるDDSP~(Differentiable~Digital~Signal~Processing)~\cite{DDSP}は同様の音色の変換において二次元データを用いることでこれらの問題を解決している。