%追加すべき事項
% - ? DDSPの簡単な説明
\chapter{まとめ}

本章では、実験結果の考察と展望についてまとめる。

\section{実験結果のまとめ}

提案手法の目標は音の高さと大きさを維持したままでギターの単音からハープの単音への音色の変換を行うことである。実験の結果、この目標に足る表現力を持つ提案モデルを作成できたが、作成した提案モデルの汎化能力は十分でないことも確認できた。

汎化能力の低い理由としては主に三点あると考えられる。一点目は、Pix2pixで用いられるDropout層を用いていない点である。Dropout層はGANのノイズを表すだけでなく、汎化能力を高める効果があると期待される。二点目は、データ拡張の工夫が振幅の無作為化のみである点であり、時間方向にずらすなどの工夫もすべきであった。三点目は、音の表現として音響信号を用いた点であり、時間-周波数表現を用いたモデルを作成して性能比較をする必要があったと考えられる。

さらに、本研究では波形の観察を中心とした考察を行ったが、本研究の考察を明確化するためにはより定量的な判定が必要であったと考えられる。具体的には、音の高さの維持、音の大きさの維持、音色の変換、という三点についての判定である。

\section{展望}

以下の三点を解決することで本研究の提案手法を音楽での音色の変換に適用できると考えられる。

\subsection{楽器の重ね合わせ}

音楽はそれぞれの楽器から出力される音の重ね合わせになっている。楽器ごとに音色は異なるため、音色の変換を行うには楽器ごとの音波に分解すること~(音源分離)~が必要である。また、楽曲の作成時に楽器ごとに分離したデータ~(パラデータ)~の公開が一般的になれば音源分離の必要はなくなる。

\subsection{音の重ね合わせ}

ある音が単音の重ね合わせである重音の場合は単音ごとに音色の変換を行う必要がある。この時、単音だけでなく重音もデータセットに加えることで音色の変換が可能であると考えられる。しかし、この際にデータセットが膨大な量になる可能性があり、追加するデータセットの工夫が必要となる。

\subsection{音の繋ぎ方}

上記の二つが可能である時、時間方向の音の繋ぎ方を工夫する必要がある。単位時間ごとに区切るのみで音色変換は可能であると考えていたが、音の連続的な音色の変化に対応するためには二次元の表現を用いるべきであると考えられる。