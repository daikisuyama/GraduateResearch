%追加すべき事項
% - データ拡張:位相をずらす、バッチサイズの調整?
\chapter{まとめ}

本研究での実験の結果、提案モデルは十分な表現力を持つことを示すことができたが、汎化能力は十分でないことも同時に明らかになった。提案モデルの生成モデルにおいてPix2pixのようなDropout層を用いていない点やデータ拡張が十分に機能しない点が汎化能力をが十分ではない主な原因であると考えられる。また、音波の細かな部分の観察により表現力も十分ではないとも推察される。そして、\prettyref{sec:result}にあるように、音の大きさの維持、音波の滑らかさの表現、音の減衰の表現、データセットの不安定さ、の四点の課題も残った。

さらに、本研究では波形の観察を中心とした考察を行ったが、本研究の考察を明確化するためにはより定量的な判定を行う必要がある。具体的には、音の高さの維持、音の大きさの維持、音色の変換、という三点での定量的な判定を行う必要がある。

そして、汎化能力の向上、課題の解決、定量的な判定法に取り組んだ後に音楽への適用を考える必要があるが、GoogleのMagentaプロジェクトにより発表されたDDSP~\cite{DDSP}は特徴量として二次元データを用いることでこれらの問題を解決している。また、DDSPの説明は\prettyref{app:DDSP}にて行う。