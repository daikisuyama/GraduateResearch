\chapter{はじめに}

音楽は世界中で楽しまれ、その作成方法は技術発展により多様化している。近年注目されている音楽の作成方法としてリミックスと呼ばれるものがある。リミックスは既存の曲に音響操作を加えて新しい曲を作成する方法であり、1970年代のディスコの発達とともに世界的に普及した。当時はディスコでのDJによる即興のパフォーマンスとして行われていたが、近年のパソコンなどのデジタル機器の発達により音楽の作成方法として一般的なものとなった。

しかし、音の編集などを行うソフトウェアアプリケーション~(DAW)~の操作がリミックスには必要であり、音楽作成を円滑に行うためには一定の経験が必要となる。したがって、コンピュータプログラムによるリミックスの補助が音楽作成に役立つと考えられる。また、本研究では、リミックスの方法の一つである音色の変換に注目し、ニューラルネットワークによる音色の変換手法を提案する。

そして、本研究では、画像のスタイル変換を行うPix2pix~\cite{pix2pix}を元に作成した提案モデルを用いてギターの単音からハープの単音へ音色を変換する実験を行った。その結果、音の高さを維持したまま変換を行うことができたが、ハープの音を表現できたのは一部の音のみであり、音色の変換におけるいくつかの課題が浮かび上がった。

なお、本論文では、音のデータセットを作成する際に楽譜作成ソフトのMuseScore\footnote{\url{https://musescore.org/}}を利用し、音波の波形画像を作成する際に音声編集ソフトのAudacity\footnote{\url{https://www.audacityteam.org/}}を利用した。