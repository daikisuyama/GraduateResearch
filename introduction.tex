%全体の流れ
%本章では…
%「この研究では」をここでまとめる
%1.1以降は用語の説明のみにとどめるようにする
%また、この研究ではニューラルネットワークを用いて音楽をremixすることを目標とするが、簡単のためにremixを音楽の構造を変化させない音色の変換による加工に限定する。
%まず、音波、人間の耳への聞こえ方
%次に音波,remix,音色の変換…


\section{Remix}

%再構成とは?
%加工とは?
%バージョンとは?
音楽を作成する方法の一つとしてremixがある。remixとは既存の曲の再構成及び加工を行うことでその曲の新しいバージョンとする方法のことである。

%音の説明
%音の三要素の説明
%その前に一つの図の中に三要素を示す




\section{音色}

%同じ音の大きさ及び高さでも人間の耳には異なって聞こえる
%これを音色と呼ぶ(二つの波形)
%また、この研究では、同じ音の大きさ及び高さで異なる種類の楽器による出力の違いを音色の違いとして捉えて実験を行う。

音色とは、日本産業規格の定義によれば「聴覚に関する音の属性の一つで、物理的に異なる二つの音が、たとえ同じ音の大きさ及び高さであっても異なった感じに聞こえるとき、その相違に対応する属性」とある。


%ここで書くのは語句の説明にとどめる
\section{音色の変換}

%なぜ音色の変換がしたいのか↓
%音色の返還の際には次のような難しさがある
%三つの要素に分解することで音楽の音色の変換を単音の音色の変換に帰着することができる
%単音による変換の話はここではしない
%単音の提案手法を書いてから提案手法2に書く

音色の変換を音楽で行う際に(音の特徴を学習する)が、以下の三つの点を解決するのが難しいと考えられる。また、以下の三つを解決することで、単音における音色の変換を音楽に適用することができる。

\subsection{楽器の重ね合わせ}

楽器ごとに音色が異なるので、楽器ごとの音波に分解して音色変換を行うことが良いと考えられる。なお、楽曲の作成時に楽器ごとに分離したデータ~(パラデータ)~で保存しておけば、直接楽器ごとの音波を利用できる。


\subsection{時間方向の音の繋ぎ方}

%ここは後で
\textcolor{red}{時間方向での音の繋ぎ方は都合の良いように分割していくことでなんとかなるのではないか…?、分割する(1つの音の判定を行う)のは難しい…?、自己回帰?}

\subsection{音の重ね合わせ}

単位時間の楽器の音に注目した時、楽器ごとの音波に分離したとしても和音のようにその単位時間で複数の種類の音が鳴っている場合も難しいと考えられる。

\textcolor{red}{とりあえず試しでも良いので実験を行いたい}

%第二章に書く

\section{既存研究}

