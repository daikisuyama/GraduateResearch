\chapter{用語の説明}

本章では、音楽用語の説明を行う。


\section{音}

弾性体~(空気)~中を伝播する弾性波により起こされる音波が聴覚により感じられるもののことである。また、音波に周期性があり明確な音程を持つ音として聞こえる場合は楽音と呼ばれる。

\section{楽音の三要素}

楽音は高さ,大きさ,音色の三つの要素~(音の三要素)~から成り立っているとされる。図1の通り、高さは音波の振動数により決まり、大きさは音波の振幅により決まる。

\subsection{楽音の音色}

音の高さと大きさが同じであるが異なった音として知覚される時の属性のことである。図2では音の高さと大きさが同じギターの音とハープの音の波形を示しており、この波形の違いが音色の違いとなる。


%Remixの説明はいらない

%\section{Remix}

%再構成とは?
%加工とは?
%バージョンとは?
%音楽を作成する方法の一つとしてremixがある。remixとは既存の曲の再構成及び加工を行うことでその曲の新しいバージョンとする方法のことである。
%また、この研究ではニューラルネットワークを用いて音楽をremixすることを目標とするが、簡単のためにremixを音楽の構造を変化させない音色の変換による加工に限定する。




\chapter{音楽の既存研究}


長期的な構造と短期的な構造のいずれもを変換しようとしており、その多くでは長期的な構造を音楽性を持ったまま変換するのが難しい(←要参考文献)

長期的な構造は別のアルゴリズムを使うのが適切では?(文章の生成などの自然言語処理に近い?)

短期的な構造のみを変えることに注目したい→音色による変換

