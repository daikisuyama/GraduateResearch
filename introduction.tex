\chapter{初めに}

%ニューラルネットはここでなんとなく説明しておく!

%概要で書きたかったこと
%全体的なことを話し始める
%お気持ち表明

本研究ではニューラルネットワークを用いて音色を変換することを目標とする。


%Remixは"初めに"に無理矢理書くのが良い
%語句の説明自体は軽く


%\section{Remix}

%再構成とは?
%加工とは?
%バージョンとは?
%音楽を作成する方法の一つとしてremixがある。remixとは既存の曲の再構成及び加工を行うことでその曲の新しいバージョンとする方法のことである。
%また、この研究ではニューラルネットワークを用いて音楽をremixすることを目標とするが、簡単のためにremixを音楽の構造を変化させない音色の変換による加工に限定する。

%ここはチラッと入れる

\section{着想}
%なぜ単音の変換

%既存研究的なことも書く
%研究の特徴にさらっと触れる感じかな

長期的な構造と短期的な構造のいずれもを変換しようとしており、その多くでは長期的な構造を音楽性を持ったまま変換するのが難しい(←要参考文献)

長期的な構造は別のアルゴリズムを使うのが適切では?(文章の生成などの自然言語処理に近い?)

短期的な構造のみを変えることに注目したい→音色による変換
