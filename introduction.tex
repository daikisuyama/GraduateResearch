\chapter{初めに}

本研究ではニューラルネットワークを用いて音楽を変換することを目標とする。変換は音楽におけるremixと呼ばれる手法と同様に行う。remixとは既存の曲の再構成及び加工を行ってその曲の新しいバージョンを作成する方法のことである。また、本研究では、remixを音楽の構造を変化させない音色の変換による加工に限定する。

\begin{comment}
%ここから
%短期vs長期
%長期は生成しておいて短期は後から
%音楽の研究の例は関連研究に
既存研究の軽い紹介…。GAN,Pix2pix…。音楽の変換の研究(Hukebox,スペクトログラム,MIDI)…。この手法では…。音色の変換のみを扱うことで短期的な構造のみに着目できる点で他の音楽生成の研究よりも計算時間を削減できると期待される。
~\cite{Jukebox}
\end{comment}