\chapter{はじめに}

音楽は世界中で楽しまれている。音楽の作成方法には様々なものがあり、既存の曲をアレンジして新しい曲を作成するRemixと呼ばれる方法がある。しかし、Remixは画面上で音を操作するソフトウェアアプリケーション~(DAW)~を用いるのが一般的である。初心者がDAWを使用するのは難しいため、コンピュータプログラムによる補助が役に立つと考えられる。本研究では、Remixの代表的な方法の一つである音色の変換に着目し、プログラムによる変換手法を提案する。

音色の変換を行うためには、ある楽器の音を異なる楽器の音へ変換する技術が必要である。そこで、本研究ではPix2pix~\cite{pix2pix}を音色の変換に応用した。Pix2pixはニューラルネットワークにより自然な画像を生成する手法であるGenerative~Adversarial~Networks~(GAN)~\cite{GAN}を用いて画像のスタイル変換を行う手法である。

本研究では、ギターの単音をハープの単音に提案手法を用いて変換する実験を行った。その結果、音の大きさが変わることなどの問題はあったものの、ほとんどの音で音程を維持したまま音色の変換を行うことに成功した。また、データセットの一部の音のみで学習を行った場合でも、ほとんどの音で音程を維持したまま変換を行うことができた。

なお、本論文では、音は楽譜作成ソフトのMuseScore\footnote{\url{https://musescore.org/}}により作成し、音波の画像は音声編集ソフトのAudacity\footnote{\url{https://www.audacityteam.org/}}により作成した。

\begin{comment}
%ここから
%短期vs長期
%長期は生成しておいて短期は後から
%音楽の研究の例は関連研究に
既存研究の軽い紹介…。GAN,Pix2pix…。音楽の変換の研究(Hukebox,スペクトログラム,MIDI)…。この手法では…。音色の変換のみを扱うことで短期的な構造のみに着目できる点で他の音楽生成の研究よりも計算時間を削減できると期待される。
~\cite{Jukebox}
\end{comment}