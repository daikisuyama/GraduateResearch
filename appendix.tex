\appendix
\chapter{}
\label{sec:appendix}

%勾配降下法のパラメータを増やす
%Adamを使った
%torch.optim.Adam
%ハイパーパラメータ(beta1,beta2,learning_rate)
\section{実験時のパラメータ}
\label{sec:appendix_params}

実験時のパラメータを表\ref{tab:params}に示す。

\begin{table}[h]
\label{tab:params}
\caption{}
\begin{center}
    \begin{tabular}{ll}\toprule
        パラメータ & 値 \\ \midrule
        インプットのバッチサイズ & 1 \\ 
        学習でのエポック数 & 1000 \\ 
        学習での学習率 & 0.0002 \\ \bottomrule
    \end{tabular}
\end{center}
\end{table}

\section{データセットの区分}
\label{sec:appendix_split}

生成モデルの汎化能力を調べた際のデータセットの区分を表\ref{tab:split}に示す。国際の階名表記に従っている。

%音階順でソート(アルファベット)
\begin{table}[h]
\label{tab:split}
\caption{}
\begin{center}
    \scalebox{0.7}{
    \begin{tabular}{l|llllllllllllllllllllll}\toprule
        番号  \\ \midrule
        0 & A5$\sharp$ & G2 & E6 & F6 & G3 & C1 & A2$\sharp$ & F7 & F5 & C2$\sharp$ & B5 & E4 & B6 & D4$\sharp$ & A4 & B7 & E1 & G4$\sharp$ & F4 & A7$\sharp$ & G1 & F2$\sharp$ \\ 
        1 & A2 & D1 & G1$\sharp$ & G4 & D5$\sharp$ & D6 & A3$\sharp$ & A1 & A5 & C6$\sharp$ & F1$\sharp$ & C1$\sharp$ & A1$\sharp$ & C5 & D2$\sharp$ & G7 & C2 & E7 & B2 & G7$\sharp$ & F5$\sharp$ & D1$\sharp$ \\ 
        2 & E3 & G5 & C7$\sharp$ & D4 & G5$\sharp$ & D6$\sharp$ & G6 & C8 & C4 & C4$\sharp$ & A0$\sharp$ & A3 & D3 & D5 & B0 & A6 & A6$\sharp$ & F7$\sharp$ & C5$\sharp$ & D3$\sharp$ & F3 & D7$\sharp$ \\ 
        3 & B4 & F3$\sharp$ & F6$\sharp$ & E2 & B1 & E5 & F2 & C3 & A4$\sharp$ & F1 & C6 & A7 & D2 & G6$\sharp$ & G2$\sharp$ & G3$\sharp$ & C3$\sharp$ & F4$\sharp$ & A0 & D7 & B3 & C7 \\ \bottomrule
    \end{tabular}
    }
\end{center}
\end{table}