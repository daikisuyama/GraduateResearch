\appendix
\chapter{}
\label{sec:appendix}

\section{実験時のパラメータ}
\label{sec:appendix_params}

実験時のパラメータを表\ref{tab:params}に示す。

\begin{table}[h]
\label{tab:params}
\caption{}
\begin{center}
    \begin{tabular}{|ll|}\hline
        パラメータ & 値 \\ \hline \hline
        インプットのバッチサイズ & 1 \\ 
        学習でのエポック数 & 1000 \\ 
        学習での学習率 & 0.0002 \\ \hline
    \end{tabular}
\end{center}
\end{table}

\section{データセットの分割方法}
\label{sec:appendix_split}

生成モデルの汎化能力を調べた際のデータセットの分割の仕方を表\ref{tab:split}に示す。

\begin{table}[h]
\label{tab:split}
\caption{}
\begin{center}
    \scalebox{0.8}{
    \begin{tabular}{|lllllllllllllllllllllll|}\hline
        番号  \\ \hline \hline
        0 & a5s & g2 & e6 & f6 & g3 & c1 & a2s & f7 & f5 & c2s & b5 & e4 & b6 & d4s & a4 & b7 & e1 & g4s & f4 & a7s & g1 & f2s\\ 
        1 & a2 & d1 & g1s & g4 & d5s & d6 & a3s & a1 & a5 & c6s & f1s & c1s & a1s & c5 & d2s & g7 & c2 & e7 & b2 & g7s & f5s & d1s \\ 
        2 & e3 & g5 & c7s & d4 & g5s & d6s & g6 & c8 & c4 & c4s & a0s & a3 & d3 & d5 & b0 & a6 & a6s & f7s & c5s & d3s & f3 & d7s \\ 
        3 & b4 & f3s & f6s & e2 & b1 & e5 & f2 & c3 & a4s & f1 & c6 & a7 & d2 & g6s & g2s & g3s & c3s & f4s & a0 & d7 & b3 & c7 \\ \hline
    \end{tabular}
    }
\end{center}
\end{table}