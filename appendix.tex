\appendix
\chapter{}
\label{sec:appendix}

%勾配降下法のパラメータを増やす
%Adamを使った
%torch.optim.Adam
%ハイパーパラメータ(beta1,beta2,learning_rate)
\section{実験時のパラメータ}
\label{sec:appendix_params}

実験時のパラメータを表\ref{tab:params}に示す。

\begin{table}[h]
\label{tab:params}
\caption{}
\begin{center}
    \begin{tabular}{lr}\toprule
        パラメータ & 値 \\ \midrule
        インプットのバッチサイズ & 1 \\ 
        学習でのエポック数 & 1000 \\ 
        学習での学習率 & 0.0002 \\ \bottomrule
    \end{tabular}
\end{center}
\end{table}

\section{データセットの分割方法}
\label{sec:appendix_split}

生成モデルの汎化能力を調べた際のデータセットのサブセットへの分割を表\ref{tab:split}に示す。表記は国際の階名表記に従い、音の高さの昇順に並んでいる。データセットの分割方法としては88音をシャッフルして配列に格納した後に22音ずつ順に選んでいる。また、4分割検証を行っているため、番号$i$~($0 \leqq i \leqq 4$)の22音のサブセットを評価に用いた際は番号$j$~($0 \leqq i \leqq 4$かつ$i \neq j$)のサブセットの66音を全て学習に用いている。

\begin{table}[h]
\label{tab:split}
\caption{}
\begin{center}
    \scalebox{0.7}{
    \begin{tabular}{l|llllllllllllllllllllll}\toprule
        番号  \\ \midrule
        0 & C1 & E1 & G1 & C2$\sharp$ & F2$\sharp$ & G2 & A2$\sharp$ & G3 & D4$\sharp$ & E4 & F4 & G4$\sharp$ & A4 & F5 & A5$\sharp$ & B5 & E6 & F6 & B6 & F7 & A7$\sharp$ & B7\\ 
        1 & C1$\sharp$ & D1 & D1$\sharp$ & F1$\sharp$ & G1$\sharp$ & A1 & A1$\sharp$ & C2 & D2$\sharp$ & A2 & B2 & A3$\sharp$ & G4 & C5 & D5$\sharp$ & F5$\sharp$ & A5 & C6$\sharp$ & D6 & E7 & G7 & G7$\sharp$ \\ 
        2 & A0$\sharp$ & B0 & D3 & D3$\sharp$ & E3 & F3 & A3 & C4 & C4$\sharp$ & D4 & C5$\sharp$ & D5 & G5 & G5$\sharp$ & D6$\sharp$ & G6 & A6 & A6$\sharp$ & C7$\sharp$ & D7$\sharp$ & F7$\sharp$ & C8 \\ 
        3 & A0 & F1 & B1 & D2 & E2 & F2 & G2$\sharp$ & C3 & C3$\sharp$ & F3$\sharp$ & G3$\sharp$ & B3 & F4$\sharp$ & A4$\sharp$ & B4 & E5 & C6 & F6$\sharp$ & G6$\sharp$ & C7 & D7 & A7 \\ \bottomrule
    \end{tabular}
    }
\end{center}
\end{table}