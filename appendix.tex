\appendix
\chapter{}
\label{sec:appendix}

\section{実験時のパラメータ}

%パラメータ,値
\begin{table}[h]
\begin{center}
    \begin{tabular}{|l||l|}\hline
        パラメータ & 値 \\ \hline \hline
        インプットのバッチサイズ & 1 \\ \hline
        学習でのエポック数 & 1000 \\ \hline
        学習での学習率 & 0.0002 \\ \hline
    \end{tabular}
\end{center}
\end{table}

\section{実験日時}

\subsection{生成モデルの表現力}

\noindent
開始:\\
終了:

\subsection{生成モデルの汎化能力}

\noindent
開始:2021-01-09 22:34:23\\
終了:2021-01-10 05:48:49

\section{データセットの分割方法}

\subsection{生成モデルの汎化能力}

\noindent
[['a5s', 'g2', 'e6', 'f6', 'g3', 'c1', 'a2s', 'f7', 'f5', 'c2s', 'b5', 'e4', 'b6', 'd4s', 'a4', 'b7', 'e1', 'g4s', 'f4', 'a7s', 'g1', 'f2s'], ['a2', 'd1', 'g1s', 'g4', 'd5s', 'd6', 'a3s', 'a1', 'a5', 'c6s', 'f1s', 'c1s', 'a1s', 'c5', 'd2s', 'g7', 'c2', 'e7', 'b2', 'g7s', 'f5s', 'd1s'], ['e3', 'g5', 'c7s', 'd4', 'g5s', 'd6s', 'g6', 'c8', 'c4', 'c4s', 'a0s', 'a3', 'd3', 'd5', 'b0', 'a6', 'a6s', 'f7s', 'c5s', 'd3s', 'f3', 'd7s'], ['b4', 'f3s', 'f6s', 'e2', 'b1', 'e5', 'f2', 'c3', 'a4s', 'f1', 'c6', 'a7', 'd2', 'g6s', 'g2s', 'g3s', 'c3s', 'f4s', 'a0', 'd7', 'b3', 'c7']]