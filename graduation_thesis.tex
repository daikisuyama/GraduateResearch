%jarticleの時はpart,section,subsection,subsubsection
\documentclass[dvipdfmx]{jarticle}
\usepackage{graphicx}
\usepackage{url}
\usepackage{float}
\usepackage{hyperref}
\usepackage{amsmath}
\usepackage{amssymb}
\usepackage{algpseudocode}
\usepackage{fancyhdr}
\usepackage{bookmark}

\title{CycleGANよる音楽の自動変換}

\author{陶山大輝 \\
所属:教養学部後期課程学際科学科総合情報学コース\\
学籍番号:08-192021\\
指導教員:金子知適准教授\\}

\begin{document}

\maketitle

%改ページ
\newpage

\begin{abstract}
\documentclass[10pt,a4paper,dvipdfmx]{jsreport}
\pagestyle{plain}
\usepackage{graphicx}
\usepackage{url}
\usepackage{float}
\usepackage{hyperref}
\usepackage{amsmath}
\usepackage{amssymb}
\usepackage{algpseudocode}
\usepackage{fancyhdr}
\usepackage{bookmark}
\usepackage{color}
\usepackage{comment}
\usepackage{booktabs}
\setcounter{secnumdepth}{3}
\newcommand{\argmax}{\mathop{\rm arg~max~}\limits}
\newcommand{\argmin}{\mathop{\rm arg~min~}\limits}

\begin{document}

\begin{center}
{\huge 論文要旨}\\
\vspace{40truept}
{\huge ニューラルネットワークによる音色の自動変換}
\end{center}

\vspace{30truept}

\begin{flushright}
{\Large 学際科学科~総合情報学コース}\\ 
\vspace{5truept}
{\Large 08-192021}\\ 
\vspace{5truept}
{\Large 陶山大輝}\\
\vspace{5truept} 
{\Large 指導教員~金子知適}\\
\end{flushright}

\vspace{30truept}

{\Large 音楽において音色の自動変換をニューラルネットワークにより行うのは難しいが、本研究ではまずPix2pixを用いた単音における音色の変換手法を提案する。また、単音における変換手法は音楽へと応用することが可能であると筆者は考えており、その応用方法についても本論文にて紹介する。}

\end{document}
%概要

\end{abstract}

\tableofcontents

%改ページ
\newpage

\part{はじめに}
\include{put}{introduction.tex}
%背景
%自分の考えも

\part{背景}
\include{previous_research.tex}
%ニューラルネットワークとは
%ディープラーニングとは
%GANとは
%pix2pixとは
%CycleGAN(検討)
%Wavenet(検討)
%スペクトログラム(検討)

\part{データセット}
楽譜作成ソフトのMuseScoreを利用して国際の階名表記でA0\~C8の音をwav形式で52音生成した。また、この52音は一般的な88鍵のピアノで出すことのできる全音階の音であり、その音域は人間が音程として聞き分けることのできる限界の音域として選んだ。\par
%https://www.yamaha.com/ja/musical_instrument_guide/piano/trivia/trivia007.html
ここで、52音については人間の耳で聴き分けられる程に十分に音色が異なると考えられるエレキギターとハープを選んだ。\par
%近い音の楽器も選んで比較するべきでは?アコースティックギター?
そして、それぞれの音については四分音符を生成したのち1秒の長さに揃えている。これらの音は全てサンプル周波数が44.1kHzで量子化ビットは16ビットである。

\subsection{データ拡張}
先程の52音のみではデータ数が十分ではないのでデータ拡張を行った。具体的には正規化したデータを$\alpha(0 < \alpha <1)$倍した後に$[0,1-\alpha)$の一様乱数を加えることにした。また、今回は$\alpha=0.01$に固定した。
%ここで、頑健性の評価に使えることを言えるとなお良い



%データセット(musescore)

\part{評価}
\include{evaluation.tex}
%評価用のプログラムを用意する

\part{提案手法}
%ここからは提案手法

\section{データ形式}
音のファイル形式としてはWAVを用いる。一般的にはMP3やMP4などの非可逆圧縮形式が広く用いられるが、WAVは波形データを直接保持しており本論文ではWAVを扱いやすいと考えたためである。

%WAVの持つ情報:A,B,CがあるがAは割愛する、みたいな説明

%離散化の際のデジタル信号の単位時間あたりの標本化の回数をサンプリング周波数と呼び、量子化の際のデジタル信号の大きさを表現するビット数を量子化ビットと呼ぶ。

%また、音は時間を定義域としたアナログ信号なので、離散化及び量子化を行ってデジタル信号へと変換することでデジタル機械の扱えるデータ形式となる。

% subsectionでメタ情報の説明
% itemizeでも

音のデータそのものだけでなく量子化ビット,サンプリング周波数,チャンネル数,サンプリング数のメタ情報を得ることができる。

また、サンプリング周波数を44100Hz,量子化ビットを16bit,チャンネル数を1,サンプリング数を44100に固定して今回の実験を行った。

\section{データセット}

楽譜作成ソフトの\href{https://musescore.org/ja}{MuseScore}を利用して国際の階名表記でA0からC8に含まれる半音をwav形式で88音生成した。また、この88音は一般的な88鍵のピアノで出すことのできる全音階の音であり、人間が音程として聞き分けることのできる限界の音域としてこの音域を選んだ。


%他の楽器との間との比較も?
%全く異なる楽器や近い楽器
%どこにかくか、表現
また、楽器としてはエレキギターとハープを選んだ。


\section{判定器}

(1)音程が維持されているか

(2)変換されて音色が変換されているか

波形とスペクトログラム?


%---------

%単音の変換手法


%それができたらどうなるかをここで書く


\section{音色の変換}

%なぜ音色の変換がしたいのか↓
%音色の返還の際には次のような難しさがある
%三つの要素に分解することで音楽の音色の変換を単音の音色の変換に帰着することができる

音色の変換を音楽で行う際に(音の特徴を学習する)が、以下の三つの点を解決するのが難しいと考えられる。また、以下の三つを解決することで、単音における音色の変換を音楽に適用することができる。

\subsection{楽器の重ね合わせ}

楽器ごとに音色が異なるので、楽器ごとの音波に分解して音色変換を行うことが良いと考えられる。なお、楽曲の作成時に楽器ごとに分離したデータ~(パラデータ)~で保存しておけば、直接楽器ごとの音波を利用できる。


\subsection{時間方向の音の繋ぎ方}

%ここは後で
\textcolor{red}{時間方向での音の繋ぎ方は都合の良いように分割していくことでなんとかなるのではないか…?、分割する(1つの音の判定を行う)のは難しい…?、自己回帰?}

\subsection{音の重ね合わせ}

単位時間の楽器の音に注目した時、楽器ごとの音波に分離したとしても和音のようにその単位時間で複数の種類の音が鳴っている場合も難しいと考えられる。

\textcolor{red}{とりあえず試しでも良いので実験を行いたい}
%手法
%結果
%考察

\part{まとめ}
%やったこと、やってないことを書く
\chapter{まとめ}


\section{future work}

%書くならfuture workみたいなところに


%それができたらどうなるか
%future research?
\subsection{音楽の変換}

%なぜ音色の変換がしたいのか↓
%音色の返還の際には次のような難しさがある
%三つの要素に分解することで音楽の音色の変換を単音の音色の変換に帰着することができる

音色の変換を音楽で行う際に(音の特徴を学習する)が、以下の三つの点を解決するのが難しいと考えられる。また、以下の三つを解決することで、単音における音色の変換を音楽に適用することができる。

\subsubsection{楽器の重ね合わせ}

楽器ごとに音色が異なるので、楽器ごとの音波に分解して音色変換を行うことが良いと考えられる。なお、楽曲の作成時に楽器ごとに分離したデータ~(パラデータ)~で保存しておけば、直接楽器ごとの音波を利用できる。


\subsubsection{時間方向の音の繋ぎ方}

%ここは後で
\textcolor{red}{時間方向での音の繋ぎ方は都合の良いように分割していくことでなんとかなるのではないか…?、分割する(1つの音の判定を行う)のは難しい…?、自己回帰?}

\subsubsection{section}{音の重ね合わせ}

単位時間の楽器の音に注目した時、楽器ごとの音波に分離したとしても和音のようにその単位時間で複数の種類の音が鳴っている場合も難しいと考えられる。

\textcolor{red}{とりあえず試しでも良いので実験を行いたい}



\subsection{判定器}

(1)音程が維持されているか

(2)変換されて音色が変換されているか

波形とスペクトログラム?


%自己回帰モデル,スペクトログラム

\subsection{和音の手法}

二音のみの組み合わせでできるのか
ネットワークを変えるべきか
ネットワークをアップデートする必要があるのか
前処理を工夫するか
フーリエ変換を用いてsin波に分解するか
%結論
%展望

\part{謝辞}
%優先順位低め

\part{参考文献}

\bibliography{ref}
\bibliographystyle{junsrt}

\end{document}