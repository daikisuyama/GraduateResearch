%注意
%図を適宜加える
%数式はalignを使う
%数式は自分の都合の良いように書き換える
%図などはデザインの本を読む
%半角括弧の前に~を置く
%用語は省略しないで書く
%使ったツールなどはURLも添付する

%jarticleの時はchapter,section,subsection,subsubsection
\documentclass[dvipdfmx]{jreport}
\usepackage{graphicx}
\usepackage{url}
\usepackage{float}
\usepackage{hyperref}
\usepackage{amsmath}
\usepackage{amssymb}
\usepackage{algpseudocode}
\usepackage{fancyhdr}
\usepackage{bookmark}
\usepackage{color}

\title{ニューラルネットワークによる\\ 音楽の自動変換}

\author{陶山大輝 \thanks{東京大学教養学部後期課程学際科学科総合情報学コース} \thanks{学籍番号:08-192021} \thanks{指導教員:金子知適准教授}}

\begin{document}

\maketitle

%タイトルページをいじると良い

\begin{abstract}
本研究ではニューラルネットワークを用いて音色を変換することを目標とする。
\end{abstract}

\tableofcontents



\chapter{はじめに}

音楽は世界中で楽しまれ、その作成方法は技術発展により多様化している。近年注目されている音楽の作成方法としてリミックスと呼ばれるものがある。リミックスは既存の曲に音響操作を加えて新しい曲を作成する方法であり、1970年代のディスコの発達とともに世界的に普及した。当時はディスコでのDJによる即興のパフォーマンスとして行われていたが、近年のパソコンなどのデジタル機器の発達により音楽の作成方法として一般的なものとなった。

しかし、音の編集などを行うソフトウェアアプリケーション~(DAW)~の操作がリミックスには必要であり、音楽作成を円滑に行うためには一定の経験が必要となる。したがって、コンピュータプログラムによるリミックスの補助が音楽作成に役立つと考えられる。また、本研究では、リミックスの方法の一つである音色の変換に注目し、ニューラルネットワークによる音色の変換手法を提案する。

そして、本研究では、画像のスタイル変換を行うPix2pix~\cite{pix2pix}を元に作成した提案モデルを用いてギターの単音からハープの単音へ音色を変換する実験を行った。その結果、音の高さを維持したまま変換を行うことができたが、ハープの音を表現できたのは一部の音のみであり、音色の変換におけるいくつかの課題が浮かび上がった。

なお、本論文では、音のデータセットを作成する際に楽譜作成ソフトのMuseScore\footnote{\url{https://musescore.org/}}を利用し、音波の波形画像を作成する際に音声編集ソフトのAudacity\footnote{\url{https://www.audacityteam.org/}}を利用した。
%自分の考え

\chapter{背景}
\section{ニューラルネットワークとは}
ニューラルネットワークとはニューロンとニューロン間のシナプスによる結合で形成される脳のネットワークを模した数理モデルのことである。入力層と出力層を持ち、シナプスの結合強度を変化させることで問題に最適なネットワークを構成することを目標とする。\par
また、入力層と出力層の間に隠れ層を加えて多層にし層間に活性化関数を用いて非線形分離を行うことで、複雑なネットワークを表現をすることが可能になる。そして、任意の活性化関数が微分可能であれば、誤差逆伝播法により損失関数を高速に求めることができる。\par


\section{ディープラーニングとは}

%ディープラーニングとは
%GANとは
%pix2pixとは
%CycleGAN(検討)
%Wavenet(検討)
%スペクトログラム(検討)
%内容はbackground.texに


%データセット、あれば評価方法も
\chapter{提案手法}
%ここからは提案手法

\section{データ形式}
音のファイル形式としてはWAVを用いる。一般的にはMP3やMP4などの非可逆圧縮形式が広く用いられるが、WAVは波形データを直接保持しており本論文ではWAVを扱いやすいと考えたためである。

%WAVの持つ情報:A,B,CがあるがAは割愛する、みたいな説明

%離散化の際のデジタル信号の単位時間あたりの標本化の回数をサンプリング周波数と呼び、量子化の際のデジタル信号の大きさを表現するビット数を量子化ビットと呼ぶ。

%また、音は時間を定義域としたアナログ信号なので、離散化及び量子化を行ってデジタル信号へと変換することでデジタル機械の扱えるデータ形式となる。

% subsectionでメタ情報の説明
% itemizeでも

音のデータそのものだけでなく量子化ビット,サンプリング周波数,チャンネル数,サンプリング数のメタ情報を得ることができる。

また、サンプリング周波数を44100Hz,量子化ビットを16bit,チャンネル数を1,サンプリング数を44100に固定して今回の実験を行った。

\section{データセット}

楽譜作成ソフトの\href{https://musescore.org/ja}{MuseScore}を利用して国際の階名表記でA0からC8に含まれる半音をwav形式で88音生成した。また、この88音は一般的な88鍵のピアノで出すことのできる全音階の音であり、人間が音程として聞き分けることのできる限界の音域としてこの音域を選んだ。


%他の楽器との間との比較も?
%全く異なる楽器や近い楽器
%どこにかくか、表現
また、楽器としてはエレキギターとハープを選んだ。


\section{判定器}

(1)音程が維持されているか

(2)変換されて音色が変換されているか

波形とスペクトログラム?


%---------

%単音の変換手法


%それができたらどうなるかをここで書く


\section{音色の変換}

%なぜ音色の変換がしたいのか↓
%音色の返還の際には次のような難しさがある
%三つの要素に分解することで音楽の音色の変換を単音の音色の変換に帰着することができる

音色の変換を音楽で行う際に(音の特徴を学習する)が、以下の三つの点を解決するのが難しいと考えられる。また、以下の三つを解決することで、単音における音色の変換を音楽に適用することができる。

\subsection{楽器の重ね合わせ}

楽器ごとに音色が異なるので、楽器ごとの音波に分解して音色変換を行うことが良いと考えられる。なお、楽曲の作成時に楽器ごとに分離したデータ~(パラデータ)~で保存しておけば、直接楽器ごとの音波を利用できる。


\subsection{時間方向の音の繋ぎ方}

%ここは後で
\textcolor{red}{時間方向での音の繋ぎ方は都合の良いように分割していくことでなんとかなるのではないか…?、分割する(1つの音の判定を行う)のは難しい…?、自己回帰?}

\subsection{音の重ね合わせ}

単位時間の楽器の音に注目した時、楽器ごとの音波に分離したとしても和音のようにその単位時間で複数の種類の音が鳴っている場合も難しいと考えられる。

\textcolor{red}{とりあえず試しでも良いので実験を行いたい}
%手法
%結果
%考察

\chapter{まとめ}
%やったこと、やってないことを書く
\chapter{まとめ}


\section{future work}

%書くならfuture workみたいなところに


%それができたらどうなるか
%future research?
\subsection{音楽の変換}

%なぜ音色の変換がしたいのか↓
%音色の返還の際には次のような難しさがある
%三つの要素に分解することで音楽の音色の変換を単音の音色の変換に帰着することができる

音色の変換を音楽で行う際に(音の特徴を学習する)が、以下の三つの点を解決するのが難しいと考えられる。また、以下の三つを解決することで、単音における音色の変換を音楽に適用することができる。

\subsubsection{楽器の重ね合わせ}

楽器ごとに音色が異なるので、楽器ごとの音波に分解して音色変換を行うことが良いと考えられる。なお、楽曲の作成時に楽器ごとに分離したデータ~(パラデータ)~で保存しておけば、直接楽器ごとの音波を利用できる。


\subsubsection{時間方向の音の繋ぎ方}

%ここは後で
\textcolor{red}{時間方向での音の繋ぎ方は都合の良いように分割していくことでなんとかなるのではないか…?、分割する(1つの音の判定を行う)のは難しい…?、自己回帰?}

\subsubsection{section}{音の重ね合わせ}

単位時間の楽器の音に注目した時、楽器ごとの音波に分離したとしても和音のようにその単位時間で複数の種類の音が鳴っている場合も難しいと考えられる。

\textcolor{red}{とりあえず試しでも良いので実験を行いたい}



\subsection{判定器}

(1)音程が維持されているか

(2)変換されて音色が変換されているか

波形とスペクトログラム?


%自己回帰モデル,スペクトログラム

\subsection{和音の手法}

二音のみの組み合わせでできるのか
ネットワークを変えるべきか
ネットワークをアップデートする必要があるのか
前処理を工夫するか
フーリエ変換を用いてsin波に分解するか
%結論
%展望

\chapter*{謝辞}
%優先順位低め
%ここに直接書く


\bibliography{ref}
\bibliographystyle{junsrt}

\end{document}