%やったこと、やってないことを書く
\chapter{まとめ}


\section{future work}

%書くならfuture workみたいなところに


%それができたらどうなるか
%future research?
\section{音楽の変換}

%なぜ音色の変換がしたいのか↓
%音色の返還の際には次のような難しさがある
%三つの要素に分解することで音楽の音色の変換を単音の音色の変換に帰着することができる

音色の変換を音楽で行う際に(音の特徴を学習する)が、以下の三つの点を解決するのが難しいと考えられる。また、以下の三つを解決することで、単音における音色の変換を音楽に適用することができる。

\subsection{楽器の重ね合わせ}

楽器ごとに音色が異なるので、楽器ごとの音波に分解して音色変換を行うことが良いと考えられる。なお、楽曲の作成時に楽器ごとに分離したデータ~(パラデータ)~で保存しておけば、直接楽器ごとの音波を利用できる。


\subsection{時間方向の音の繋ぎ方}

%ここは後で
\textcolor{red}{時間方向での音の繋ぎ方は都合の良いように分割していくことでなんとかなるのではないか…?、分割する(1つの音の判定を行う)のは難しい…?、自己回帰?}

\subsection{音の重ね合わせ}

単位時間の楽器の音に注目した時、楽器ごとの音波に分離したとしても和音のようにその単位時間で複数の種類の音が鳴っている場合も難しいと考えられる。

\textcolor{red}{とりあえず試しでも良いので実験を行いたい}



\subsection{判定器}

(1)音程が維持されているか

(2)変換されて音色が変換されているか

波形とスペクトログラム?


%自己回帰モデル,スペクトログラム