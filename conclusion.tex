%やったこと、やってないことを書く
%実験で見つかった課題とかも
\chapter{まとめ}

本研究での実験の結果、単音での音色の変換を行うにはニューラルネットワークのモデルの工夫,学習時のデータの振幅の乱雑さの加え方の工夫,安定したデータセットの作成の三点が特に必要であることがわかった。

また、4分割交差検証を行った時のデータの区分はランダムであるためにバランスの良いデータセットにした時の検証など追加実験が必要であることもわかった。

さらに、単音の変換ができた後の音楽への適用についてもまだ取り組めておらず、単音の音色変換を完全に行うことができた後に取り組むべきである。

そして、本研究では波形の観察による考察を行ったが、より定量的な判定を行うには判定器の導入が必要であり、音程が維持されているかと音色が正しく変換されているかの二つの方向性での判定器が必要である。前者については周波数スペクトルから求めることができるが、後者については自分で構成する必要があると考えている。