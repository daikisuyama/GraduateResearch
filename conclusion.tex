\chapter{まとめ}
%結果の概要をまとめる(トピックセンテンス)

%追加実験が必要
%次の和音の変換にも取り組む
本研究での実験の結果、提案手法における生成モデルはギターからハープへと音色の変換を行うには十分な表現力を持つことを確認することができた。ただし、ニューラルネットワークのモデルによる細かい音波の表現、学習時のデータの振幅の乱雑さの加え方、安定したデータセットの作成、の三つの点においては課題が残った。そのため、4分割交差検証を行った際にはその影響が増幅し、音程はほとんどの変換で維持できているものの、ハープの音へと変換することができたのは一部のみであった。

また、本研究では波形の観察による考察を行ったが、より定量的な判定を行う必要がある。具体的には、音程が維持されているかと音色が正しく変換されているかの判定を考慮できると良い。前者についてはフーリエ変換を用いて実装することができるが、後者については考察の余地がある。

\section{課題}

%結果の課題をまとめる

\section{展望}

一般に、音楽で特定の音色への変換を行うことは難しいが、次の三つの要素に分解することで単音での音色の変換を音楽へと適用することが可能であると考えられる。また、三つの要素とは、楽器の重ね合わせ、音の重ね合わせ、時間方向の音の繋ぎ方であり、それぞれについて具体的に以下で説明をする。

\subsection{楽器の重ね合わせ}
    
音楽はそれぞれの楽器により奏でられた音の重ね合わせになっている。楽器ごとに音色は異なるので楽器ごとの音波に分解して音色変換を行う~(音源分離)~が必要であると考えられる。なお、楽曲の作成時に楽器ごとに分離したデータ~(パラデータ)~で保存しておけば、音源分離を行わずに直接楽器ごとの音波を利用できる。

\subsection{音の重ね合わせ}

ある単位時間の音に注目した時、楽器ごとの音波に分解してもその単位時間で異なった高さや大きさの音の重ね合わせになっていることがある。この場合は、今回の提案手法で用意したデータセット以外に和音のデータセットも加えて学習させることで表現可能であると考えられる。

\subsection{音の繋ぎ方}

%他の工夫があるかも

楽器ごとの音波に分解し単位時間の音が表現できた時、時間方向に音を繋ぐ必要がある。時間方向については先程定めた単位時間で区切って順に変換していくことで可能であると 考えている。また、区切るのみでの変換が難しい場合は自己回帰モデルを取り入れるなどの工夫が必要であると考えている。

%既存ではアナログからデジタルへの変換で、こことの違いは?