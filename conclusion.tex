\chapter{まとめ}

%追加実験が必要
%次の和音の変換にも取り組む
本研究での実験の結果、提案手法における生成モデルはギターからハープへと音色の変換を行うには十分な表現力を持つことを確認することができた。ただし、ニューラルネットワークのモデルによる細かい音波の表現、学習時のデータの振幅の乱雑さの加え方、安定したデータセットの作成、の三つの点においては課題が残った。そのため、4分割交差検証を行った際にはその影響が増幅し、音程はほとんどの変換で維持できているものの、ハープの音へと変換することができたのは一部のみであった。

また、本研究では波形の観察による考察を行ったが、より定量的な判定を行うには判定器の導入が必要である。判定器は音程が維持されているかの判定と音色が正しく変換されているかの判定のいずれも行うことができると良い。前者についてはフーリエ変換を用いて実装することができ、後者については考察の余地がある。