%やったこと、やってないことを書く
%実験で見つかった課題とかも
\chapter{まとめ}

%ネットワーク変えれるのでは?
%もっとランダムにできるのでは?
本研究では〇〇を確認することができた。以後は以下をやる。

\section{今後の研究}

\subsection{音楽における音色の変換}

本研究では単音における実験を行ったが、これを音楽で用いるには以下の三つの点を解決するのが難しいと考えられる。

\subsubsection{楽器の重ね合わせ}

楽器ごとに音色が異なるので、楽器ごとの音波に分解して音色変換を行うことが良いと考えられる。なお、楽曲の作成時に楽器ごとに分離したデータ~(パラデータ)~で保存しておけば、直接楽器ごとの音波を利用できる。


\subsubsection{時間方向の音の繋ぎ方}

時間方向での音の繋ぎ方は都合の良いように分割していくことでなんとかなるのではないか…?、分割する(1つの音の判定を行う)のは難しい…?、自己回帰モデル必要か?

\subsubsection{音の重ね合わせ}

単位時間の楽器の音に注目した時、楽器ごとの音波に分離したとしても和音のようにその単位時間で複数の種類の音が鳴っている場合も難しいと考えられる。

二音のみの組み合わせでできるのか
前処理を工夫するか
フーリエ変換を用いてsin波に分解するか

意外とできそう

\subsection{判定器}

本研究では波形の観察による考察を行ったが、より定量的な判定を行うには判定器の導入が必要であると考えられ、具体的には以下の二点での判定が必要である。

(1)音程が維持されているか

(2)変換されて音色が変換されているか

波形とスペクトログラム?