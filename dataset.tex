\section{音}
弾性体中を伝播する弾性波により起こされる音波が聴覚されるもののこと。この論文では弾性体として空気のみを想定する。

また、音は時間を定義域としたアナログ信号なので、離散化及び量子化を行ってデジタル信号へと変換することでデジタル機械の扱えるデータ形式となる。



%ここからは提案手法

\section{データ形式}
音のファイル形式としてはWAVを用いる。一般的にはMP3やMP4などの非可逆圧縮形式が広く用いられるが、WAVは波形データを直接保持しており本論文ではWAVを扱いやすいと考えたためである。

%WAVの持つ情報:A,B,CがあるがAは割愛する、みたいな説明

%離散化の際のデジタル信号の単位時間あたりの標本化の回数をサンプリング周波数と呼び、量子化の際のデジタル信号の大きさを表現するビット数を量子化ビットと呼ぶ。

% subsectionでメタ情報の説明
% itemizeでも

音のデータそのものだけでなく量子化ビット,サンプリング周波数,チャンネル数,サンプリング数のメタ情報を得ることができる。

また、サンプリング周波数を44100Hz,量子化ビットを16bit,チャンネル数を1,サンプリング数を44100に固定して今回の実験を行った。

\section{データセット}

楽譜作成ソフトの\href{https://musescore.org/ja}{MuseScore}を利用して国際の階名表記でA0からC8に含まれる半音をwav形式で88音生成した。また、この88音は一般的な88鍵のピアノで出すことのできる全音階の音であり、人間が音程として聞き分けることのできる限界の音域としてこの音域を選んだ。


%他の楽器との間との比較も?
%全く異なる楽器や近い楽器
%どこにかくか、表現
また、楽器としてはエレキギターとハープを選んだ。