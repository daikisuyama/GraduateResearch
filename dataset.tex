楽譜作成ソフトのMuseScoreを利用して国際の階名表記でA0\~C8の音をwav形式で52音生成した。また、この52音は一般的な88鍵のピアノで出すことのできる全音階の音であり、その音域は人間が音程として聞き分けることのできる限界の音域として選んだ。\par
%https://www.yamaha.com/ja/musical_instrument_guide/piano/trivia/trivia007.html
ここで、52音については人間の耳で聴き分けられる程に十分に音色が異なると考えられるエレキギターとハープを選んだ。\par
%近い音の楽器も選んで比較するべきでは?アコースティックギター?
そして、それぞれの音については四分音符を生成したのち1秒の長さに揃えている。これらの音は全てサンプル周波数が44.1kHzで量子化ビットは16ビットである。

\subsection{データ拡張}
先程の52音のみではデータ数が十分ではないのでデータ拡張を行った。具体的には正規化したデータを$\alpha(0 < \alpha <1)$倍した後に$[0,1-\alpha)$の一様乱数を加えることにした。また、今回は$\alpha=0.01$に固定した。
%ここで、頑健性の評価に使えることを言えるとなお良い


