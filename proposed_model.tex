\chapter{提案モデル}
\label{chap:pr_model}

本章では、本研究で用いたニューラルネットワークモデルを説明する。

\section{ニューラルネットワークのモデル}

本研究では、Pix2pixを音楽でも用いることができるように改変した。
%二次元データではなく一次元データである点
%既存ではアナログからデジタルへの変換で

\subsection{GANのモデル}

%ここを書く

\subsection{生成モデル}

生成モデルには、

カーネルサイズが3,パディングが1の5層の畳み込みネットワークを用いた。

\subsection{識別モデル}

識別モデルには、カーネルサイズが4,パディングが4の5層の畳み込みネットワークを用いた。出力層を除き、各層の出力には活性化関数としてLeaky~ReLU関数を用いた。また、初めの3つの層ではストライドを2に、後半の二つの層ではストライドを1にした。