\documentclass[10pt,a4paper,dvipdfmx]{jsreport}
\pagestyle{plain}
\usepackage{graphicx}
\usepackage{url}
\usepackage{float}
\usepackage{hyperref}
\usepackage{amsmath}
\usepackage{amssymb}
\usepackage{algpseudocode}
\usepackage{fancyhdr}
\usepackage{bookmark}
\usepackage{color}
\usepackage{comment}
\usepackage{booktabs}
\setcounter{secnumdepth}{3}
\newcommand{\argmax}{\mathop{\rm arg~max~}\limits}
\newcommand{\argmin}{\mathop{\rm arg~min~}\limits}

\begin{document}

\begin{center}
{\huge 論文要旨}\\
\vspace{40truept}
{\huge ニューラルネットワークによる音色の自動変換}
\end{center}

\vspace{30truept}

\begin{flushright}
{\Large 学際科学科~総合情報学コース}\\ 
\vspace{5truept}
{\Large 08-192021}\\ 
\vspace{5truept}
{\Large 陶山大輝}\\
\vspace{5truept} 
{\Large 指導教員~金子知適}\\
\end{flushright}

\vspace{30truept}

{\Large 音楽において音色の自動変換をニューラルネットワークにより行うのは難しいが、本研究ではまずPix2pixを用いた単音における音色の変換手法を提案する。また、単音における変換手法は音楽へと応用することが可能であると筆者は考えており、その応用方法についても本論文にて紹介する。}

\end{document}