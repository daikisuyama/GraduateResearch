%実験は過去形に統一
\chapter{提案手法}

本章では、本研究での提案手法である単音の音色の変換について紹介する。音楽の音色の変換については、以下の三点を解決することで今回の提案手法を適用することができると考えられる。

\begin{description}
%ここの中身は途中

\item[楽器の重ね合わせ]\mbox{}

楽器ごとに音色が異なるので、楽器ごとの音波に分解して音色変換を行うことが良いと考えられる。なお、楽曲の作成時に楽器ごとに分離したデータ~(パラデータ)~で保存しておけば、直接楽器ごとの音波を利用できる。

\item[時間方向の音の繋ぎ方]\mbox{}

時間方向での音の繋ぎ方は都合の良いように分割していくことでなんとかなるのではないか…?、分割する(1つの音の判定を行う)のは難しい…?、自己回帰モデル必要か?

\item[音の重ね合わせ]\mbox{}

単位時間の楽器の音に注目した時、楽器ごとの音波に分離したとしても和音のようにその単位時間で複数の種類の音が鳴っている場合も難しいと考えられる。

\end{description}

\section{データセット}

本研究のデータセットの作成方法及びその形式についての説明を行う。

\subsection{データセットの作成方法}

楽譜作成ソフトのMuseScore\footnote{\url{https://musescore.org/}}により国際の階名表記でA0からC8までの半音を88音生成した。これらの音は88鍵のピアノで出すことのできる音であり、最も一般的な音域として今回の実験では選んだ。また、ギターからハープの音へと音色の変換を行うので、そのどちらも88音を生成した。

\subsection{データ形式}

音のファイル形式としては非圧縮形式のWAV形式を用いる。MP3形式やMP4形式などの非可逆圧縮形式も一般には広く用いられるが、音波の波形データを直接保持しているために扱いやすいWAV形式を本論文で用いることにした。また、WAV形式は波形データ以外にメタデータを持ち、本論文で扱うメタデータについて以下で説明をする。

\begin{description}

\item[サンプリング周波数]\mbox{}

サンプリング周波数とは、デジタル信号の1秒あたりの標本化の回数のことである。本論文では44100Hzに固定して実験を行う。

\item[サンプリング数]\mbox{}

サンプリング数とは、デジタル信号の標本化の合計の回数のことである。本論文では44100回に固定して実験を行う。

\item[量子化ビット数]\mbox{}

量子化ビット数とは、デジタル信号の細かさを表現するビット数のことである。本論文では16bitに固定して実験を行う。

\item[チャンネル数]\mbox{}

チャンネル数とは、モノラルな音声の出力の総数のことである。本論文では1に固定して実験を行う。

\end{description}

\section{手法}

Pix2pixを用いて十分に音色が異なると考えられるギターからハープへの音の変換を行うことを目標に実験を行った。また、学習とテストの際のパラメータは付録\ref{sec:appendix}の\ref{sec:appendix_params}節に示す。

\subsection{ニューラルネットワークのモデル}

Pix2pixにおいては以下のようなニューラルネットワークのモデルを用いた。

%全体,generator,discriminatorの順に書く
%できれば図があると良い


\subsection{生成モデルの表現力}

生成モデルの表現力を測るために学習データとテストデータに同じ88音のデータセットを用いて実験を行った。また、過学習を防ぐために各エポックでそれぞれのデータの振幅を$c \in [0.3,1]$倍して学習を行った。

\subsection{生成モデルの汎化能力}

生成モデルの表現力が十分にあることを確認したので、その汎化能力を調べる実験を行った。また、先程の実験と同様に各エポックでそれぞれのデータの振幅を$c \in [0.3,1]$倍して学習を行った。

さらに、汎化能力を調べるために、88音のデータに対し4分割交差検証を行った。この際にデータはランダムに分割しており、その区分は付録\ref{sec:appendix}の\ref{sec:appendix_split}に示した。

\section{実験結果}

本節では、実験結果及びその考察をまとめる。また、生成された音波の波形を画像で添付する際に音声編集ソフトのAudacity\footnote{\url{https://www.audacityteam.org/}}を用いた。

\subsection{生成モデルの表現力}

%学習の際のlossは図〇〇のようになった(軽く)
%ここで結果と考察

\subsection{生成モデルの汎化能力}


%学習の際のlossは図〇〇のようになった(軽く)
%ここで結果と考察

%仮説:周波数を勝手に理解できるのでは?

