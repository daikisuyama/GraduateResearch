\chapter{提案手法}

本研究で使用したデータセット及び提案手法の説明を行う。

\section{データセット}

本研究のデータセットの作成方法及びその形式についての説明を行う。

\subsection{データセットの作成方法}

楽譜作成ソフトのMuseScore\footnote{\url{https://musescore.org/}}により国際の階名表記でA0からC8までの半音をwav形式で88音生成した。また、これらの音は88鍵のピアノで出すことのできる音であり、最も一般的な音域として今回の実験では選んだ。

%他の楽器との間との比較も?:全く異なる楽器や近い楽器
そして、音色の変換を行う楽器としては十分に音色が異なると考えられるエレキギターとハープを選んだ。

\subsection{データ形式}

音のファイル形式としては非圧縮形式のWAVを用いる。MP3やMP4などの非可逆圧縮形式も一般には広く用いられるが、音波の波形データを直接保持しているために扱いやすいWAVを本論文で用いることにした。また、WAVの持つ波形データ以外のメタデータのうち本論文で扱うものについて以下で説明をする。

\begin{description}

\item[サンプリング周波数]\mbox{}

サンプリング周波数とは、デジタル信号の単位時間あたりの標本化の回数のことである。本論文では44100Hzに固定して実験を行う。

\item[サンプリング数]\mbox{}

サンプリング数とは、デジタル信号の標本化の合計の回数のことである。本論文では44100回に固定して実験を行う。

\item[量子化ビット数]\mbox{}

量子化ビット数とは、デジタル信号の大きさを表現するビット数のことである。本論文では16bitに固定して実験を行う。

\item[チャンネル数]\mbox{}

チャンネル数とは、モノラルな音声の出力の総数のことである。本論文では1に固定して実験を行う。

\end{description}

\section{手法}

%pix2pixを使った
%この時に変えたパラメータなどを列挙する

\section{実験}

\subsection{生成モデルの表現力}

%結果、考察

\subsection{生成モデルの汎化能力}

%結果、考察
