\chapter{音楽}

本章では、音楽用語の定義を行った後に、音のデータの主要な前処理の方法を紹介する。

\section{音}

音とは、弾性体~(空気)~中を伝播する弾性波により起こされる音波が聴覚により感じられるもののことである。また、音は騒音と楽音の大きく二つに分けることができる。騒音とは不規則な振動の音波による音のことであり、楽音とは周期生のある音波による音のことである。本論文では、楽音のことを音と呼ぶ。

\section{音の要素}

音は長さ,大きさ,高さ,音色の四つの要素から成り立つ~\cite{音楽の基礎}。人間はこれらの四つの要素を知覚することができる。

\begin{figure}[t]
\begin{center}
\includegraphics[width=\hsize]{figure/gakuon.png}
\caption{音の要素}
\label{fig:gakuon}
\end{center}
\end{figure}

\subsection{音の長さ}

音の長さは図\ref{fig:gakuon}のように音波の時間の長さにより決まる。一般に、音の長さは楽譜上での時間の長さ~(音価)~により決まるが、本論文では音波の時間の長さにより決まるものとする。

\subsection{音の大きさ}

音の大きさは図\ref{fig:gakuon}のように音波の振幅により決まる。また、人間には振幅の大きい音は大きく、振幅の小さい音は小さく知覚される。

\subsection{音の高さ}

音の高さは音波の周波数により決まる。つまり、図\ref{fig:gakuon}のような音の周期的な構造の長さにより決まる。また、人間には周波数の高い音は高く、周波数の低い音は低く知覚さる。そして、複数の周波数の音波が音に含まれる場合は最も低い周波数成分の音波~(基音)~を音の高さとして知覚する。

国際の音名表記では、C,D,E,F,G,A,Bのセットを西洋音楽の七音音階におけるオクターブとして定める。さらに、それぞれのオクターブに番号を振り、440~Hzの音をA4と定めることで、任意の半音の絶対的な表記を可能にしている。また、国際の音名表記では半音よりさらに細かい音~(微分音)~を表すことはできないが、本論文では扱わない。

\subsection{音の音色}

音の長さと高さと大きさが同じであっても異なった音として人間には知覚されることがある。この違いを音色と呼ぶ。また、音色は図\ref{fig:gakuon}のような音の周期的な構造の形により決まる。

\begin{figure}[t]
\begin{center}
\includegraphics[width=\hsize]{figure/c4_guitar_harp.png}
\caption{ギターとハープの音色}
\label{fig:guitar_harp_comp}
\end{center}
\end{figure}

ここで、長さと高さと大きさが同じギターとハープの音波を図\ref{fig:guitar_harp_comp}に示す。これらの音は波形が異なるので、その音色も異なる。また、音色の異なる音どうしは基音よりも高い音~(上音)~の組み合わせが異なるため、このような波形の違いが生まれる。

\section{音声処理}
\label{sec:preprocess}

%音声処理勉強してここに載せる(大幅な改変)

%以下の流れ

%まず基本的なこと

%既存研究でもこんな感じ

%それぞれに特徴があるけど…

\begin{comment}
本章では、主要な音声処理の方法を紹介する。

以下の単語は説明

サンプリング周波数] デジタル信号の1秒あたりの標本化の回数

サンプリング数] 音波のデジタル信号の標本化の合計の回数のこと

量子化ビット数] デジタル信号の細かさを表現するビット数のこと

チャンネル数] モノラルな音声出力の総数のこと

%短期vs長期
%長期は生成しておいて短期は後から
%音楽の研究の例はをどこかに
既存研究の軽い紹介…。GAN,Pix2pix…。音楽の変換の研究(Hukebox,スペクトログラム,MIDI)…。この手法では…。音色の変換のみを扱うことで短期的な構造のみに着目できる点で他の音楽生成の研究よりも計算時間を削減できると期待される。
\end{comment}